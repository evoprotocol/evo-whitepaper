%% Copyright 2020 All Rights Reserved
%% SPDX-License-Identifier: GNU General Public License v2.0 only
%% 
%% This program is free software; you can redistribute it and/or modify it under the terms of the GNU General Public License as published by the Free Software Foundation; version 2.
%% This program is distributed in the hope that it will be useful, but WITHOUT ANY WARRANTY; without even the implied warranty of MERCHANTABILITY or FITNESS FOR A PARTICULAR PURPOSE. See the GNU General Public License for more details.
%% You should have received a copy of the GNU General Public License along with this program; if not, write to the Free Software Foundation, Inc., 51 Franklin Street, Fifth Floor, Boston



%% EVO %% 
\chapter{Introduction}\label{sec:introduction}

EVO Protocol presents an idea of embedded volumetric optionality, which in essence is a mechanism for controllable liquidity through the mechanism of volume. Any asset would be "less" accelerating in price swings depending on the chose parameters. One consequence is that significant volume movements, such as "rug pulls" or legitimate price action movements, incur a cost onto the liquidating party. This, in essence, ensures orderly market liquidation to a degree with an additional benefit of helping provide an orderly unwinding / unwinding period for instruments utilizing the protocol. Instruments may produce predictable inflation rates, enabling them to be lent out through other protocols, furthering market stabilizing effects. Such token can be a reasonable investment choice purposed for the "storage of value".

\section{Abstract}\label{sec:preamble}
A Generalized Protocol Specification for Volumetric Manifolds and a reference implementation. Features include an embedded 
volumetric mechanism to enforce desired behaviors
based upon robust economic incentives.\footnote{An analogy to help understand these terms is the \textbf{ManifoldSystem}. The \textbf{ManifoldEpoch} is equivalent to the Epoch,
the \textbf{ManifoldVolume} is equivalent to the
transfer rate, and the \textbf{ManifoldFee} is the
Fee for the Transfer Rate.}

%\lstset{language=TeX,numbers=none}
%\begin{lstlisting}[frame=lines]
\begin{verbatim}
Key Concepts: Automatic Stabilizers, embedded volumetric
optionality, liquidity, transaction unwinding, 
transaction execution, price stability, self stability
 \end{verbatim}
%\end{lstlisting}
    \vspace{2mm}

\subsection{Transaction Unwinding: Risk and Execution and Settlement}

How does selling against an order book work? How do you measure risk of not being able to liquidate your position? Any trader (esp. low volume
low liquidity) can tell you how a tin order book will make it 
impossible to profit on a trade. We can look at Almgren's Optimal 
Execution of Portfolio Transactions\footnote{Almgren 1998, https://www.math.nyu.edu/faculty/chriss/optliq_f.pdf} on their analysis of the situation:
    \vspace{2mm}
\subsection{Trade Half Life}
This analysis yields a number we call the "half-life" of a trade, the 
natural time for execution in the absence of exogenous time 
constraints. For example, in trading a highly illiquid, volatile 
security, there are two extreme strategies: trade everything now at a 
known, but high cost, or trade in equal sized packets over $x$ time 
at a relatively lower cost. The latter strategy has lower expected cost but this comes at the expense of greater uncertainty in total revenue.
    \vspace{2mm}
\subsection{Optimal Execution of Portfolio Transactions}
    \vspace{2mm}
\subsection{Front Running and Backrunning}
This can be expressed as the common "Gas Wars" events of network 
congestion on ethereum foundation, or sending in a transaction knowingly under priced in the expectation it will be included in a block in $x$ amount of time from now. Basically discounting a future 
block inclusion as an acceptable cost versus having it included relatively sooner (i.e. you would rather spend less get it later than 
spend more and have it now).

$Twait$\text{discounted future block inclusion }
    \vspace{2mm}
$Twait$\text{ discounted future block inclusion}
    \vspace{2mm}
\equation{$$Tunwind$$ time to liquidate position}



We can define this as what we call the time it takes to unwind a position (i.e. liquidation period)

    \vspace{2mm}

All we can do is insist that for a given level uncertainty, cost be minimized.We can say that trade execution is the "Dynamic trading
strategy that provides the minimum expected cost of trading over $x$
period of time"\footnote{Almgren 1998}. As such we can say that the
strategy employed for transaction pricing in the first auction system
can be that of expectations of future physical settlement (i.e.
transaction being included in the block) versus willingness to pay for
that inclusion (i.e bidding expressed in transaction fees).
    \vspace{2mm}
\subsection{Volumetric Contracting}

A Volumetric futures contract would be buying a contract for a certain
volume of goods, such as freight space, gas (i.e. natural gas
products), or $block space$. Block space as a tradable volumetric
futures contract could be achieved by creating a set of contracts in
which gas token can either be re-packaged, re-sold (auction), or burned and minted.
    \vspace{2mm}
\subsection{Establishing Parameters for Transfer Rate}

\begin{tabular}{llrr}
\hline Strategy & Parameter & Gas price & Blocks waited \\
\hline Geth & \(S =1.0\) & 4,414,902,746 & 1.97 \\
Geth & \(S =0.9\) & 4,080,968,868 & 15.49 \\
Geth & \(S =0.8\) & 3,531,922,197 & 25.52 \\
\hline Look-ahead & \(B =15\) & 1,166,965,099 & 4.80 \\
Look-ahead & \(B =30\) & 969,559,938 & 8.52 \\
Look-ahead & \(B =60\) & 782,105,012 & 18.84 \\
\hline Proposed approach & \(U =1.0\) & 2,120,108,703 & 3.28 \\
Proposed approach & \(U =0.9\) & 1,908,097,833 & 3.79 \\
Proposed approach & \(U =0.8\) & 1,696,086,963 & 5.13 \\
Proposed approach & \(U =0.7\) & 1,484,076,092 & 10.06
\end{tabular}
    \vspace{2mm}
\subsection{Contract Delivery and Contract Rollover Dates}
    \vspace{2mm}
We can extrapolate the contract for date of delivery as being subsets of $B$ and be able to 
approximate ideal delivery time for contracts. To the extent of future pricing concerns could 
express themselves beyond 30 days is  likely unrealistic at this stage. 
    \vspace{2mm}
This preliminary model is used in deciding the parameters for the first market on \textbf{evo 
protocol}, gasevo, which is $24 days$
    \vspace{2mm}
In the paper Blockchain Resource Pricing \footnote{page 14 of https://github.com/ethereum/research/raw/master/papers/pricing/ethpricing.pdf p43)} Buterin
illustrates that how under, "full block conditions, cryptocurrency price fluctuation is lower than
transaction fee fluctuations". By denominating things as "fixed fees" which utilize a unit of
account denominated in cryptocurrency, it actually leads to more (fiat-denominated) price
predictability than a market with bidding and a gas limit. In essence this is a "cap and trade" regime of monetary policy. 
    \vspace{2mm}


\quote{The regression finds that, after accounting for this social uncle rate, one byte accounts for an
additional ~0.000002 uncle rate. Bytes in a transaction take up 68 gas, of which 61 gas accounts
for its contribution to bandwidth (the remaining 7 is for bloating the history database). If we
want the bandwidth coefficient and the computation coefficient in the gas table to both reflect propagation time, then this implies that if we wanted to really 
optimize gas costs, we would need to increase the gas cost per byte by 50 (ie. to 138). This would also entail raising the base gas cost of a transaction by 5500 
(note: such a re balance would not mean that everything gets more
expensive; the gas limit would be raised by ~10% so that the average-case transaction throughput
would remain unchanged). } Vitalik Buterin, ethermagican forums. 
    \vspace{2mm}

\footnote{ One relatively simple way to implement this is that ½ of the tx fee goes to the miner
who mined the tx, and the other ½ goes into the "spread-out tx fee bucket". Each time a block is
mined, 1% of the ZEC in the "spread-out tx fee bucket" goes to the miner who mined that block.
}

